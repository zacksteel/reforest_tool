\documentclass[]{article}
\usepackage{lmodern}
\usepackage{amssymb,amsmath}
\usepackage{ifxetex,ifluatex}
\usepackage{fixltx2e} % provides \textsubscript
\ifnum 0\ifxetex 1\fi\ifluatex 1\fi=0 % if pdftex
  \usepackage[T1]{fontenc}
  \usepackage[utf8]{inputenc}
\else % if luatex or xelatex
  \ifxetex
    \usepackage{mathspec}
  \else
    \usepackage{fontspec}
  \fi
  \defaultfontfeatures{Ligatures=TeX,Scale=MatchLowercase}
\fi
% use upquote if available, for straight quotes in verbatim environments
\IfFileExists{upquote.sty}{\usepackage{upquote}}{}
% use microtype if available
\IfFileExists{microtype.sty}{%
\usepackage{microtype}
\UseMicrotypeSet[protrusion]{basicmath} % disable protrusion for tt fonts
}{}
\usepackage[margin=1in]{geometry}
\usepackage{hyperref}
\hypersetup{unicode=true,
            pdfborder={0 0 0},
            breaklinks=true}
\urlstyle{same}  % don't use monospace font for urls
\usepackage{graphicx,grffile}
\makeatletter
\def\maxwidth{\ifdim\Gin@nat@width>\linewidth\linewidth\else\Gin@nat@width\fi}
\def\maxheight{\ifdim\Gin@nat@height>\textheight\textheight\else\Gin@nat@height\fi}
\makeatother
% Scale images if necessary, so that they will not overflow the page
% margins by default, and it is still possible to overwrite the defaults
% using explicit options in \includegraphics[width, height, ...]{}
\setkeys{Gin}{width=\maxwidth,height=\maxheight,keepaspectratio}
\IfFileExists{parskip.sty}{%
\usepackage{parskip}
}{% else
\setlength{\parindent}{0pt}
\setlength{\parskip}{6pt plus 2pt minus 1pt}
}
\setlength{\emergencystretch}{3em}  % prevent overfull lines
\providecommand{\tightlist}{%
  \setlength{\itemsep}{0pt}\setlength{\parskip}{0pt}}
\setcounter{secnumdepth}{0}
% Redefines (sub)paragraphs to behave more like sections
\ifx\paragraph\undefined\else
\let\oldparagraph\paragraph
\renewcommand{\paragraph}[1]{\oldparagraph{#1}\mbox{}}
\fi
\ifx\subparagraph\undefined\else
\let\oldsubparagraph\subparagraph
\renewcommand{\subparagraph}[1]{\oldsubparagraph{#1}\mbox{}}
\fi

%%% Use protect on footnotes to avoid problems with footnotes in titles
\let\rmarkdownfootnote\footnote%
\def\footnote{\protect\rmarkdownfootnote}

%%% Change title format to be more compact
\usepackage{titling}

% Create subtitle command for use in maketitle
\providecommand{\subtitle}[1]{
  \posttitle{
    \begin{center}\large#1\end{center}
    }
}

\setlength{\droptitle}{-2em}

  \title{}
    \pretitle{\vspace{\droptitle}}
  \posttitle{}
    \author{}
    \preauthor{}\postauthor{}
    \date{}
    \predate{}\postdate{}
  

\begin{document}

{
\setcounter{tocdepth}{2}
\tableofcontents
}
\hypertarget{top}{\section{Best Management Practices for climate-wise
reforestation}\label{top}}

This document outlines some recent advances in reforestation strategies
in an age of climate change and altered disturbance regimes. We rely
heavily on the 2019
\href{https://www.sciencedirect.com/science/article/pii/S0378112718313161?via\%3Dihub}{\emph{Tamm
Review: Reforestation for resilience in dry western U. S. forests}} by
North et al, which readers are encouraged to review for further details.
A research brief of the article is also available \href{}{here}. For
brevity, conventional reforestation practices are not covered in this
document. It is our hope that practitioners will incorporate these state
of the science strategies with previously established forestry knowledge
and tailor solutions for each unique project.

\begin{center}\rule{0.5\linewidth}{\linethickness}\end{center}

\subsection{Table of Contents}\label{table-of-contents}

1 - \protect\hyperlink{Link1}{Seed Zonation}

2 - \protect\hyperlink{Link2}{Seedling Density and Spatial Arrangement}

3 - \protect\hyperlink{Link3}{Species Composition}

4 - \protect\hyperlink{Link4}{Prescribed Burning in Young Stands}

5 - \protect\hyperlink{Link5}{Future Site Suitability}

6 - \protect\hyperlink{Link6}{Example Reforestation Scenario}

7 - \protect\hyperlink{Link7}{Potential Ecological Benefits}

8 - \protect\hyperlink{Link8}{Further Reading}

\begin{center}\rule{0.5\linewidth}{\linethickness}\end{center}

\hypertarget{Link1}{\subsection{1 - Seed Zonation}\label{Link1}}

We suggest areas of recent drought- or wildfire-caused tree mortality be
subdivided into three management zones:

\begin{enumerate}
\def\labelenumi{\arabic{enumi})}
\item
  Areas where natural tree recruitment is likely to be successful and
  active reforestation is unnecessary. Specifically aforested lands near
  existing seed sources. As a course rule of thumb, areas within 650ft
  (200m) of live trees can be designated as zone 1. However, field (
  \href{https://esajournals.onlinelibrary.wiley.com/doi/full/10.1002/ecs2.1609}{Welch
  et al. 2016} ) and geospatial (
  \href{https://esajournals.onlinelibrary.wiley.com/doi/10.1002/eap.1756}{Shive
  et al. 2018} ) tools can be used to more precisely identify where
  natural recruitment will most likely be successful.
\item
  Areas where natural recruitment will likely be sparse or non-existent
  and where active reforestation should be focused. Zone 2 can be
  defined as aforested lands outside of zone 1, where existing roads and
  moderate topography allow for efficient reforestation, and where
  forests are the desired ecological state.
\item
  Stands which would otherwise be included in zone 2 but where
  reforestation is prohibitively costly due to remoteness or topography,
  or where non-forested states are desirable.
\end{enumerate}

 \textbf{Tamm Review Figure 3.} \emph{``A partially salvaged area two
years after the 2014 Eiler Fire near Burney, California. Zone 1,
outlined in green, indicates areas likely to receive seed from adjacent
islands of green trees. Zone 2, in the remaining area beyond most
natural recruitment, are the areas readily accessible for reforestation.
Two areas within this zone, A and B separated by the blue dashed line,
indicate gentler, more uniform topography (A) and more variable, steeper
sloped conditions (B), each of which could have a different planting
strategy discussed in the text. The unsalvaged, snag area in the center
could be planted if safety allows (facilitating future forest habitat
connectivity) or left to provide wildlife habitat for post-fire
specialists. Zone 3, outlined in red in the distant center of the photo,
is a steep slope, distant from access roads that might be planted with
founder stands (groups of seedlings in mesic, sheltered microsites less
likely to burn or become drought stressed).''}\\

\protect\hyperlink{top}{top}

\begin{center}\rule{0.5\linewidth}{\linethickness}\end{center}

\hypertarget{Link2}{\subsection{2 - Seedling Density and Spatial
Arrangement}\label{Link2}}

Plant seedlings in lower densities and with higher spatial heterogeneity
than conventional ``pines in lines'' approaches. Plantations that
emulate a pattern of individual scattered trees, clumps of trees, and
openings (ICO) are more likely to be resilient to increasing rates of
droughts and wildfires. Whenever possible, clusters of seedlings and
relatively densely planted areas should be aligned with topographic and
soil conditions that support greater available water and water holding
capacity, as well as areas of potential fire refugia.

 \textbf{Tamm Review Figure 1.} \emph{``Different tree planting patterns
compared to an `ICO' stand structure. Upper left is area burned by the
2007 Moonlight Fire seven years after the fire. The left side of the
road is private land regularly planted with ponderosa pine and treated
with herbicide. The right side of the road, U.S. Forest Service land,
was left unsalvaged and unplanted. The upper right photo is a cluster
planted area ten years after the 2006 Boulder Fire. Lower left is a
50-year old ponderosa pine plantation nearby but outside the Moonlight
and Boulder burns. The lower right photo shows the `ICO' pattern
produced by an active fire regime in an unmanaged Jeffrey pine stand in
the Sierra San Pedro del Martir, Baja, Mexico.''}\\

\protect\hyperlink{top}{top}

\begin{center}\rule{0.5\linewidth}{\linethickness}\end{center}

\hypertarget{Link3}{\subsection{3 - Species Composition}\label{Link3}}

Promote mixed-species stands dominated by drought and fire-resistant
trees such as pines, particularly in drier areas (e.g.~southwest-facing
slopes). Within active reforestation areas (i.e.~zone 2), the species
mix can be determined at the time of planting. In areas of passive
reforestation (i.e.~zone 1), managers may choose to inter-plant pines
among natural regeneration if nearby seed trees are primarily composed
of drought and fire-intolerant species (e.g.~firs and cedars). In both
active and passive reforestation areas, consider the importance of
including oaks, aspens and other hardwoods in the species mix. These
species diversify wildlife habitat and often serve as natural fuel
breaks, making them integral in overall landscape heterogeneity and
resilience.

\protect\hyperlink{top}{top}

\begin{center}\rule{0.5\linewidth}{\linethickness}\end{center}

\hypertarget{Link4}{\subsection{4 - Prescribed Burning in Young
Stands}\label{Link4}}

Introduce prescribed fire to planted stands early to emulate natural
fire return intervals of 5-10 years for yellow pine and 10-20 years for
mixed conifer forests. Emerging research suggests prescribed fire in
young stands can support multiple management objects while resulting in
relatively low levels of tree mortality. Such treatments can be used to
reducing surface fuels, maintain the evolutionary selection for
fire-resistant trees, promote stand heterogeneity, and avoid the high
costs and production of activity fuels associated with some mechanical
treatments. However, prescriptions developed for mature stands will
likely need to be adjusted to accommodate burning in young stands.

 \textbf{Tamm Review Figure 7.} \emph{``Examples of prescribed burning
in young stands on the Shasta-Trinity National Forest. The upper pair
are before (a) and after (b) photos from a mixed-conifer plantation that
was masticated and burned (in spring) 33 years after planting, showing
reduction in surface fuels and removal of some understory stems. The
lower pair are before (c) and after (d) photos from a plantation with
considerable added tree density due to natural regeneration, that was
masticated, branch pruned, and burned (in fall) 25 years after planting.
In the latter case, the prescribed fire was effectively a pre-commercial
thinning, reducing stand density closer to desired levels and also
generating within-stand spatial heterogeneity.''}\\

\protect\hyperlink{top}{top}

\begin{center}\rule{0.5\linewidth}{\linethickness}\end{center}

\hypertarget{Link5}{\subsection{5 - Future site
suitability}\label{Link5}}

When determining site suitability, consider current and future
suitability in addition to whether a site was previously forested.
Specifically, planting could be avoided in sites that are likely to be
marginal for the pre-disturbance species. Those areas with low mean
annual precipiation and high mean annual temperatures are likely to
become increasingly unsuitable with climate change. Local soil and
topographic characteristics could also be considered to determine
overall site suitability. For example, some hot and dry areas, with
shallow soils, and steep slopes could be allowed to transition to more
drought-tolerant forest species (e.g.~grey pine and oaks) or vegetation
types (e.g.~montane chaparral or native grasslands). When reforesting
with conifer species, selecting appropriate seed sources (e.g.~using the
\href{https://seedlotselectiontool.org/sst/}{seedlot selection tool}),
would improve the likelihood of successful tree establishment.

 \emph{Example of a low elevation burned landscape with advanced natural
conifer regeneration most successful in the drainages and north-facing
slopes. Photo credit: Marc Meyer}\\

\protect\hyperlink{top}{top}

\begin{center}\rule{0.5\linewidth}{\linethickness}\end{center}

\hypertarget{Link6}{\subsection{6 - Example Reforestation
Scenario}\label{Link6}}

From North et al. (2019), a theoretical planting strategy and stand
development graphic according to the recommendations highlighted here:

 \textbf{Tamm Review Fig. 6.} \emph{``Schematic of the initial planting
and stand development for a dissected, more fire and drought prone 0.2
ha (0.5 ac, 105 by 210 ft) slope of mixed-conifer forest where favorable
cluster microsites are more easily identified. (A) Initial planting
schematic (usually within 1--5 years following disturbance). First more
mesic microsites (concavities in the figure) are identified and planted
with clusters of trees and then the remaining area is planted with
individual trees on a regularly spaced grid (here 4.6 m or 15′ by 15′).
In this example only 60 of 115 (i.e., if fully planted on a 4.6 m
spacing) potential trees are regularly planted, and 22 are planted in
four clusters at mesic microsites. (B) After the first burn (15 years
after planting). In this hypothetical example, of the 82 original
conifers, eight have died over the last period and nine were killed by
the prescribed fire, reducing live tree density to 65 on the 0.2 ha (0.5
ac). The prescribed fire, designed to maintain tree and shrub
separation, has also killed some shrubs. (C) After 77 years of growth.
Fire has been applied every 15 years to reduce fuels and shrub cover. In
this example, 22 more trees have been killed by drought and prescribed
fire, leaving a mature forest density of 40 conifer and three oak live
trees (212 tree/ha or 86 trees/ac), within the estimated historical
mixed conifer density range of 59--329 tree/ha (24--133 trees/ac)
(Safford and Stevens, 2017).''}\\

\protect\hyperlink{top}{top}

\begin{center}\rule{0.5\linewidth}{\linethickness}\end{center}

\hypertarget{Link7}{\subsection{7 - Potential Ecological
Benefits}\label{Link7}}

Reforestation strategies that promote variable stand density at local
scales and lower densities at watershed scales can have multiple
benefits for ecological process and biodiversity. These potential
benefits include:

\begin{itemize}
\item
  Resistence of individual trees and resilience of a stand to future
  disturbance (wildfire, drought, pests and pathogens)
\item
  More stable carbon pools
\item
  Variable microclimates that can serve as climate refugia
\item
  Greater understory plant diversity
\item
  Greater wildlife diversity
\item
  Increased soil moisture and streamflow
\end{itemize}

 \emph{Example of a variable density stand in a frequently burned forest
in Yosemite National park. Photo credit: Marc Meyer}\\

\protect\hyperlink{top}{top}

\begin{center}\rule{0.5\linewidth}{\linethickness}\end{center}

\hypertarget{Link8}{\subsection{8 - Additional Reading \&
Resources}\label{Link8}}

\begin{itemize}
\item
  \href{https://www.sciencedirect.com/science/article/pii/S0378112718313161?via\%3Dihub}{\emph{Tamm
  Review: Reforestation for resilience in dry western U.S. forests} -
  North et al. 2019}
\item
  \href{https://esajournals.onlinelibrary.wiley.com/doi/full/10.1002/ecs2.1609}{\emph{Predicting
  conifer establishment post wildfire in mixed conifer forests of the
  North American Mediterranean‐climate zone} - Welch et al. 2016}
\item
  \href{https://esajournals.onlinelibrary.wiley.com/doi/10.1002/eap.1756}{\emph{From
  the stand scale to the landscape scale: predicting the spatial
  patterns of forest regeneration after disturbance} - Shive et al.
  2018}
\item
  \href{https://seedlotselectiontool.org/sst/}{\emph{Seedlot Selection
  Tool}}
\end{itemize}

\protect\hyperlink{top}{top}


\end{document}
